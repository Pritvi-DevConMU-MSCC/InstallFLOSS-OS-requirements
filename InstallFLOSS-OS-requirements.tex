\documentclass[12pt,openbib]{beamer}

\author{Pritvi Jheengut @zcoldplayer} %\thanks{\and}}
\title{Installation of an OS on a persistent memory system}
\subtitle{Procure a clean OS from scratch}
\subject{}

\usepackage[size=custom,width=64,height=36,orientation=landscape,
  scale=1.75]{beamerposter}

\usepackage{lmodern}
\usepackage[T1]{fontenc}
\usepackage[utf8]{inputenc}

% Eliminate errors such as
% \LaTeX\ Font Warning: Font shape `T1/cmss/m/n' in size <4> not available
% \LaTeX\ Font Warning: Size substitutions with differences up to 1.0pt

\usepackage{hyperref}
\hypersetup{
  pdfcreator=pdflatex,
  pdffitwindow=true
}

\usepackage{verbatim}

\usepackage[french,UKenglish]{babel}
\usepackage{listings}

% \usepackage{pgf}
% \usepackage{xy}[all]
\usepackage{graphicx}

\usepackage{tikz}
\usetikzlibrary{shapes.geometric, arrows,positioning}

\usepackage{smartdiagram}

% \usepackage{booktabs}

% usetheme{DevConMU2019}

\mode<presentation>

  \useoutertheme{infolines}

  \definecolor{light}{HTML}{0088FF}
  \definecolor{confblue}{HTML}{111144}
  \definecolor{confred}{HTML}{771111}
  \definecolor{confyellow}{HTML}{FF5533}

  \setbeamercolor{alerted text}
  {bg=confblue,fg=-confyellow}

  \setbeamercolor{block title}
  {bg=confred!95,fg=white}

  \setbeamercolor{block body}
  {bg=confblue!95,fg=confyellow}

  \setbeamercolor{frametitle}
  {parent=confyellow,fg=confyellow!35}

  \setbeamercolor{normal text}
  {bg=confblue,fg=confyellow}

  \setbeamercolor*{palette primary}
  {use=structure,fg=confblue,bg=structure.fg}

  \setbeamercolor*{palette secondary}
  {use=structure,fg=confblue,bg=structure.fg!85!-confblue}

  \setbeamercolor*{palette tertiary}
  {use=structure,fg=confblue,bg=structure.fg!70!-confblue}

  \setbeamercolor{structure}
  {bg=-confblue,fg=light!75}

  \setbeamercolor{Title bar}
  {fg=confred!90}

  \setbeamercolor{titlelike}
  {parent=-confyellow,bg=structure.fg!30!-confyellow,fg=black!85}

  \setbeamercovered{transparent,dynamic}

  \setbeamersize{text margin left=8cm,text margin right=2cm}

  \setbeamertemplate{blocks}[rounded][shadow=true]

  \setbeamertemplate{background canvas}[vertical shading]
  [top=confred,middle=confblue,bottom=confred!75]

  \setbeamertemplate{sidebar canvas left}{

    \vspace{6cm}

    \includegraphics[width=30cm,height=48cm]{5years.eps}

    \vspace{-32cm}

    \includegraphics[width=8.4cm,height=8cm]{mscc.eps}

  }

% \mode<handout>{\beamertemplatesolidbackgroundcolor{black!50}}

\mode<all>

\begin{document}

\section{Title : Presentation for \#DevConMU}

\frame{
  \frametitle{Developers Conference \#DevConMU \\ @ the Voila Hotel
    \&  Flying Dodo}

  \huge
  \maketitle

}

\section{}
\section{Intellectual Property Alert : Overview of the Copyleft
  Licenses}

\frame{
  \frametitle{Copyleft License Attribution}

  Made with love using beamer, \LaTeX\ and git.\\
  You can view at \href{https://github.com/Pritvi-DevConMU-MSCC/InstallFLOSS-OS-requirements}
  {Install a FLOSS OS}

  \begin{alertblock}{This work is licensed under the \LaTeX\ Project
      Public License.}
    To view a copy of this license, visit \\
    \url{https://www.latex-project.org/lppl.txt}
  \end{alertblock}

  \begin{alertblock}{This work is licensed under the Creative
      Commons Attribution-ShareAlike 4.0 International License.}
    To view a copy of this license, visit \\
    \href{http://creativecommons.org/licenses/by-sa/4.0/}{CC BY-SA}
    or \\

    send a letter to \\
    Creative Commons, \\
    PO Box 1866, \\
    Mountain View, \\
    CA 94042, \\
    USA. \\
  \end{alertblock}

}

\section{Greetings}
\subsection{About the Author}

\frame{
  \frametitle{Who Am I}

  \large

  \href{http://slackware.com/}{Geek@Slackware}

  \href{https://twitter.com/zcoldplayer}{twitter @zcoldplayer}

  \href{https://xmail.net/z.coldplayer}{zcoldplayer xmail Website}

  \href{http://metservice.intnet.mu/}{Work
    : SMTT@Meteorological.Services.mu}

  Active in many User group LUGM, MMC, MSCC, FECM, GDG\_MU \\
  and several other hackathons

  Passionate about how and why things work.

  Fervour Advocate of Free Libre and Open Source Software.

}

\subsection{About the Audience}

\frame{
  \frametitle{Who are you}

  \large

  \begin{block}{Would you mind tell me who you are?}
    Some hints:
  \end{block}

  \begin{enumerate}
  \item @twitter\_handle
  \item where you work
  \item email you want to share
  \item Hobbies
  \item purpose and expectations of this session
  \end{enumerate}

}

\section{Community Groups in Mauritius}

\frame{
  \frametitle{Community Groups in Mauritius}

  \begin{block}{Healthy growth of Community Groups in Mauritius}
    This turn of the century has seen an uprising of Community groups
    in Mauritius in the field of the Digital World. The diversity has
    helped the exchange and sharing of innovative ideas, experience,
    bleeding edge technology, upcoming events, conferences in the
    Digital Island of the Republic of Mauritius.
  \end{block}

  This is a list of some of the active communities in Digital
  Mauritius.

  \begin{itemize}
  \item Linux User Group Meta, LUGM
  \item Mauritius Software Craftsmanship Community, MSCC
  \item Mauritius Makers Community, MMC
  \item Front-End Coders Mauritius, FECM
  \item PHP User Group of Mauritius, phpMauritiusUG
  \item Symfonymu
  \item Google Developers Group Mauritius, GDG\_M
  \item Digital Marketing Mauritius
  \end{itemize}

}

\frame{
  \frametitle{About YOUR experience}

  \begin{block}{Have you ever installed a Free version of a Unix
      Operating System?}

  \end{block}
}

\frame{
  \frametitle{Jump into the fire}

  \begin{block}{After this session, it should be possible to install
      an OS yourself}
    without much ado after two demo.

    A basic knowledge of computers is expected.
  \end{block}

}

\frame{
  \frametitle{Distrowatch}

  \href{https://distrowatch.com}{Distrowatch Website}

   DistroWatch is a website dedicated to talking about, reviewing
   and keeping up to date with open source operating systems.

   \begin{block}{What is an ISO Image}

     An ISO image is a disk image of an optical disc. The name ISO
     is taken from the ISO 9660 file system used with CD-ROM media
     but might also contain an UDF file system.

   \end{block}

}

\frame{
  \frametitle{Mint Linux}

  Linux Mint is a modern, elegant and user comfortable operating
  system based on Linux Kernel, GNU component and other components
  to ensure ease to use by not sacrificing security.

  \href{https://linuxmint.com}{Linux Mint}
}

\frame{
  \frametitle{ }

  \begin{block}{What is Slackware Linux}

    The Official Release of \href{http://www.slackware.com/
    }{Slackware} Linux by \href{https://twitter.com/volkerdi?lang=en
        }{Patrick Volkerding} is an advanced Linux operating system,
      designed with the twin goals of ease of use and stability as
      top priorities.

    Including the latest popular software while retaining a sense of
    tradition, providing simplicity and ease of use alongside
    flexibility and power, Slackware brings the best of all worlds
    to the table.

  \end{block}

}

\section{end}

\frame{
  \frametitle{Questions}

  \huge

  \begin{center}

    QUESTIONS!

  \end{center}

}

\end{document}
